\documentclass{article}
\usepackage[utf8]{inputenc}
\usepackage[spanish]{babel}
\usepackage{graphicx}
\title{      \includegraphics[scale=0.5]{logo-universidad-de-antioquia.png}

GRANDES PENSADORES QUE DIERON VIDA A LA COMPUTACIÓN MODERNA}
\author{Informática 2
}

\date{\today}


\begin{document}



\maketitle 

"La matemática debe mucho a David Hilbert. Fue una mente universal y, y por tanto la Crisis de los Fundamentos atrajo su atención. Formuló un grandioso programa  el cual consistía, en primer lugar, en elaborar un método que permitiese construir la matemática en base a un conjunto de axiomas. Luego, se debe elaborar un método que pruebe la consistencia o inconsistencia de la teoría” (Ortíz, 1990).

Los veintitrés problemas que planteó Hilbert atrajeron a multitud de jóvenes investigadores a intentar dar solución a ellos, entre ellos Kurt Godel.Un estudiante de física teórica, que tras adentrarse al mundo de la lógica de Russell y asistir a las conferencias de David Hilbert sobre sus programas de completud y consistencia, decide cambiar el rumbo de su carrera y así estudiar lógica matemática.



\end{document}
